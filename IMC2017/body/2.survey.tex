\section{A Survey of NFV Testing Environment Deploying}
%To further prove our observation,
To get first-hand information about researchers' experience on
NFV testing environment deploying,
we did a survey among top conference authors
dedicated in the field of NFV.
We delivered the survey to sixteen people
who are the first authors of their papers %published on top conferences
and we finally collected eight responses.

In the first place, we asked for the number of types of VNF they used,
and the source of VNF, chaining policy and workload generator.
Averagely, six types of VNF are used in one paper.
Most of the VNF are open source implementations,
though three of our respondents have written their own VNF.
%a list of open source NFs found in papers??
The chaining policies are all find in papers or public policies.
And for the workload generator,
half of them directly used open source software,
and half of them wrote by themselves.
%need some conclusion?

Another question we concerned most is
the time they spent on deploying a NFV testing environment.
To more precisely measure the labor time,
we use the metric of man-month
which indicates the number of months used
if the work is done by one person.
We also asked for the scale of physical servers and VM instances,
as well as the virtualization technology they used.
The result is shown in Table \ref{survey}.

\begin{table}[!t]
\newcommand{\tabincell}[2]{\begin{tabular}{@{}#1@{}}#2\end{tabular}}
\centering
\begin{tabular}{|c|c|c|c|c|}\hline
& \tabincell{c}{Man-Month\\Used} & \tabincell{c}{\# of\\Servers} & \tabincell{c}{\# of\\VMs} & \tabincell{c}{Virtualization\\Technology}\\\hline
1 & $<$0.5 & 4 & 11-20 & KVM\\\hline
2 & 0.5-1 & 2 & 1-5 & Hyper-V\\\hline
3 & 0.5-1 & 4 & 1-7 & Container\\\hline
4 & 0.5-1 & 4 & 6-10 & KVM\\\hline
5 & 1-2 & 4 & 6-10 & \tabincell{c}{Container,\\KVM and other}\\\hline
6 & 4 & 10 & 100 & KVM\\\hline
7 & 3 & 24 & 72 & KVM\\\hline
8 & \tabincell{c}{Constantly\\Iterating} & 4 & 1-5 & Xen\\\hline
\end{tabular}
\caption{The result of our survey reflects the relationship
between labor consuming and the scale of the testing environment.}
\label{survey}
\end{table}

From the responses, we can see the average time
to deploy a testing environment for NFV is around 1-2 months
for a cluster not in very large scale.
It can be quite fast for a experienced person,
as the responser No.1, who used less than half a month
setting up a cluster including 4 physical servers and 11-20 VMs.
However, for the others, the deploying process
can be time taking and painful.
Take respondent No.6 and No.7 as examples,
they built their test environment in quite large scale,
and it results in much longer setting up time.

NFV setting up involves multi-layers in the whole software stack,
which includes virtualization, cluster management,
network virtualization, and NF configuration,
so that any point can become a bottleneck,
especially for a green hand.
Take respondent No.8 as an example,
Most problems they encountered are related to the hypervisor they used.
And respondent No.7 said they changed from Xen
to KVM with Openstack and things got much better.
We asked the authors for their most painful experiences,
and we summarize the reasons making the setting up time so long
as the following:

\begin{itemize}
\item[-]{}
Automating the process of setting up the testbed.
\item[-]{}
Setting up the datapath (interfaces, DPDK, routing tables, etc).
\item[-]{}
Installing proper rules in OpenFlow-enabled switches to enforce chaining.
\item[-]{}
Figure out appropriate workload to test different types of VNFs.
\item[-]{}
Figuring out how to determine that NFs were done starting up and ready to forward packets.
\end{itemize}

%Automating the process of setting up the testbed end to end took us a lot of time and many iterations.
%%No.4
%
%We wrote a bunch of scripts that simplified most of the process. The primary trouble was in figuring out how to determine that NFs were done starting up and ready to forward packets.
%
%Two parts for me:
%1. Install, configure and stabilize the open-source software, especially under heavy workload.
%2. Figure out appropriate workload to test different types of VNFs. It would be great if there is some real-world traffic/workload traces.
%
%Traffic generation and topology configuration
%
%Setting up the datapath (interfaces, DPDK, routing tables, etc)
%Installing proper rules in OpenFlow-enabled switches to enforce chaining

One of our responser pointed out that:

%\begin{itemize}
%\item[]{}
``If there is a system that will take as input key parameters (e.g., number of nodes, topology, VM images, choice of hypervisor) and then automatically generates a ready to go set up, that will be of a huge value!"
%!!!!but actually, we cannot...
%\end{itemize}

%This suggestion firms our intention to publish
The survey indicates the strong demand of an NFV benchmark
which provides typical NF services and workload generator,
automatically runs tests according to user configurations,
and reports the measurement results in the end.
In particular, the benchmark should support swift deployment 
especially for large NF scale.


