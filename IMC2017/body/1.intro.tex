\section{Introduction}

Network function virtualization (NFV) has become a hot topic, both in industry and academia. Since the publication of NFV Introductory White Paper \cite{} of ETSI in 2012, a lot of works have been emerged in this field. Modification works of NFV were done on the whole software stack (or maybe both SW and HW stack?). There are de facto industrial NFV platform OPNFV \cite{noauthor_home_nodate} as well as advanced NF allocation frameworks like OpenNF \cite{gember-jacobson_opennf:_2014}, CoMb \cite{180672} and E2 \cite{palkar_e2:_2015}. Also, there are works like OpenBox \cite{bremler-barr_openbox:_2016} and Netbricks \cite{199352} to rewrite or modify the NFs.

%Embark\cite{194934}
%StateAlyzer\cite{194932}
%Split/Merge\cite{180307}
%ClickOS\cite{179771}
%APLOMB\cite{sherry_making_2012}
%FTMB\cite{sherry_rollback-recovery_2015}
%DFC\cite{194970}

However, a suite of easy-to-use NFV benchmark is not yet existed. It is unavoidable to experience a time consuming process finding both open source software and proper chaining policies. According to our observation, most NFs used in papers are different open source implementations linked in different kinds of NF chain policies. So that, there is not yet a general baseline for measurement and comparison. Furthermore, the workload generator and network traffic trace used are also different, traces need to be provided to test different NF chains.

We also did a survey among top conference researchers who have experiences setting up NFV environment. In our survey, we found that the deploying time varies much due to the scale. In average, it takes 1-2 months to build up a NF cluster having less than 10 VM instances. But when the scale of instances increase to over 50, the build up process can take 3-4 months or more. One of our respondent said that they were still keeping on iterating and improving their testbed constantly.



In this paper, we develop a suite of NF benchmarks, which is supposed to have the following characteristics: 1) Representing typical NFs 2) Easy to use 3) Plenty metrics for measurement.

According to our research and observation,
the representative of an NFV benchmark
should be satisfied in three aspects:
%\begin{itemize}
%\begin{enumerate}
%\item{\textbf{Representative NFs.}}
\textbf{1) Representative NFs.}
For the demand of representative NFs,
we referred to the NFV Introductory White Paper \cite{},
which defined ten scenario of NFV use cases.
In the first version of our benchmark,
the open source implementation of NFs we collected
covers half of the ten use cases.
Table \ref{nfs} lists the basic information of NFs used in our benchmark.
%\item{\textbf{Representative NF chains.}}
\textbf{2) Representative NF chains.}
However, not only single NFs should be typical, but also the NF chains.
We referred to ETSI standard documents of SFC
(Service Function Chaining) \cite{draft-ietf-sfc-dc-use-cases-06}
for the typical use case of service chains
in the scenarios of both enterprise user and datacenter.
We also consulted our industrial partners for real world chaining policies.
The typical NF chains our benchmark provides are listed in Table \ref{chains}.
%\item{\textbf{Proper workload generator.}}
\textbf{3) Proper workload generator.}
Since each NF chain serves at different network level,
only one workload generator is not enough to test all the scenarios.
So we select different clients for each chain.
%and fixed the content of each chain.
%\end{itemize}
%\end{enumerate}


The goal of the `easy to use' design is to
achieve one-week setting up as well as one-click test running,
no matter the scale of the testing environment.
To finish a test, users only need to touch one configuration file
and execute one single command.
In the end, the measurement report will be output to a file.
To implement our design,
we leverage Docker and Kubernets to pack NFs in docker,
manage the images, and do allocation automatically.
We use OVS to do switching and packet force forwarding.
Pre-written scripts and Openflow rules are written
to implement chaining and flow control.
%There are more complains pointing out their pain points: 1) Automate the setting up and testing process. 2) Configure and stabilize NFs (?). 3) Write rules to set up topology and enforce flow control.
%%our design to solve the pain points????
%`easy to use' design should solve pain points

In our benchmark, we provide the most concerned metrics for measurement,
that is latency and throughput.
We measure latency in the granularity of per-packet and per-NF,
and output cumulative distribution of latency in the measurement report.

%The main contribution of our work is

%\section{Background and Motivation}
%
%\subsection{Network Function Virtualization}
%NFV involves multi-layers in the whole software stack,
%
%The concept of NFV chain